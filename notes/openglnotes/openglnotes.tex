\documentclass[UTF8]{article}
\usepackage{ctex}
\usepackage{multirow,booktabs}
\usepackage{amsmath,amsthm,amsfonts,amssymb,bm,mathrsfs,upgreek} 
\usepackage[paper=a4paper,top=3.5cm,bottom=2.5cm,
left=2.7cm,right=2.7cm,
headheight=1.0cm,footskip=0.7cm]{geometry}
\usepackage{color,graphicx,verbatim}
\RequirePackage{setspace}%%ÐÐŒäŸà
\setstretch{1.523}

\begin{document}
% built on 2020020719:45

\section{basic}

顶点数组对象:Vertex Array Object,VAO
顶点缓冲对象:Vertex Buffer Object,VBO
索引缓冲对象:Element Buffer Object,EBO或Index Buffer Object,IBO

3D坐标转为2D坐标的处理过程是由OpenGL的图形渲染管线(Graphics Pipeline)管理。
图形渲染管线可以被划分为几个阶段,每个阶段将会把前一个阶段的输出作为输入。
所有这些阶段都是高度专门化的(它们都有一个特定的函数),并且很容易并行执行。

显卡具有成千上万可并行执行的处理核心,每个核心在渲染管线的各个阶段运行各自的叫做着色器(Shader)的小程序,从而在图形渲染管线中快速处理数据。


图元(Primitive):指定数据(坐标和颜色值)的渲染类型,如是一系列点还是一系列三角形等。任何一个绘制指令的调用都将把图元传递给OpenGL。这是其中的几个:$GL_POINTS、GL_TRIANGLES、GL_LINE_STRIP$

以三维空间的一个三角形为例,输入的是三个3d点,作为顶点数据(Vertex Data)。依次经过:
1)顶点着色器(Vertex Shader),3d坐标变成另外一种3d坐标);

2)图元装配(Primitive Assembly),装配成指定的图元形状,这里是三角形;

3)几何着色器(Geometry Shader)。几何着色器把图元形式的一系列顶点的集合作为输入,它可以通过产生新顶点构造出新的(或是其它的)图元来生成其他形状。例子中,它生成了另一个三角形。

4)光栅化阶段(Rasterization Stage),这里它会把图元映射为最终屏幕上相应的像素,生成供片段着色器(Fragment Shader)使用的片段(Fragment)。在片段着色器运行之前会执行裁切(Clipping)。裁切会丢弃超出你的视图以外的所有像素,用来提升执行效率。
OpenGL中的一个片段是OpenGL渲染一个像素所需的所有数据。

5)片段着色器的主要目的是计算一个像素的最终颜色,这也是所有OpenGL高级效果产生的地方。片段着色器包含3D场景的数据(比如光照、阴影、光的颜色等等)。

6)Alpha测试和混合(Blending)阶段。最终阶段。这个阶段检测片段的对应的深度(和模板(Stencil))值,用它们来判断这个像素是其它物体的前面还是后面,决定是否应该丢弃。这个阶段也会检查alpha值(alpha值定义了一个物体的透明度)并对物体进行混合(Blend)。所以,即使在片段着色器中计算出来了一个像素输出的颜色,在渲染多个三角形的时候最后的像素颜色也可能完全不同。

对于大多数场合,只需要配置顶点和片段着色器。必须定义至少一个顶点着色器和一个片段着色器(因为GPU中没有默认的顶点/片段着色器)。几何着色器是可选的,通常按默认。







\end{document}